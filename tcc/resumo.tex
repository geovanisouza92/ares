% Resumo

\begin{resumo}
\label{tcc:resumo}
\noindent
Os sistemas de informa\ca o modernos v\^em aumentando em tamanho, incorporando mais dinamismo e alterando significativamente durante a execu\ca o para atender as necessidades e crescimento das organiza\co es que os mant\'em. Sendo assim, \eh\ necess\'ario que esses programas tenham boa efici\^encia, levando a oportunidade de incorporar an\'alises e transforma\co es ao longo de sua vida.\\\\
O LLVM \eh\ um conjunto de ferramentas para executar as an\'alises e transforma\co es, que inclui uma representa\ca o de c\'odigo intermedi\'aria e uma estrutura de gera\ca o de c\'odigo de m\'aquina.\\\\
Este trabalho prop\~oe-se a demonstrar a aplica\ca o do LLVM para constru\ca o de uma linguagem de programa\ca o independente de plataforma que usufrui da abordagem de "otimiza\ca o em m\'ultiplpos passos" proposta pela ferramenta.\\[1.5cm]
PALAVRAS-CHAVE: \Keywords
\end{resumo}
