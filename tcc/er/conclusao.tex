
\chapter{Considera\co es finais}

O projeto de um compilador por si s\oh\ j\ah\ \eh\ desafiador, mesmo utilizando ferramentas facilitadoras como Flex e Bison. O assunto, apesar de ter d\eh cadas de pesquisas e trabalhos, ainda \eh\ muito obscuro e pouco explorado, provavelmente devido a sua complexidade.

O projeto LLVM demonstra sua maturidade por seus diversos trabalhos derivados, incluindo-se tamb\eh m produtos comerciais dispon\ih veis no mercado. Sua arquitetura estimula a pesquisa e desenvolvimento de projetos de compiladores, otimiza\ca o, detec\ca o de falhas e explora\ca o de caracter\ih sticas avan\cc adas da engenharia de software, desde programas triviais at\eh\ sistemas operacionais.

A maior vantagem de se estudar compiladores e linguagens de programa\ca o, est\ah\ na possibilidade de conhecer melhor como os softwares s\ao\ constru\ih dos, melhorando pr\ah ticas de desenvolvimento e a forma de pensar nas solu\co es de software.

As plataformas de programa\ca o que disponibilizam caracter\ih sticas de auto-gerenciamento otimizam o desenvolvimento, provendo resultados melhores.
