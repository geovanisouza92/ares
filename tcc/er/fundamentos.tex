% Seção: Fundamentos de linguagens de programação

% \section{Fundamentos de linguages de programa\ca o}
% \label{revisao:fundamentos}

% \cite{Aho08} afirma que um dos aspectos mais importantes do projeto de uma linguagem de programa\ca o e de um compilador, \eh\ o conjunto de pol\'iticas que a linguagem seguir\ah.

% Isso diz respeito sobre qual poder de decis\ao e assertividade o compilador ter\ah\ para auxiliar o programador na constru\ca o de programas corretos.

% \subsection{Est\ah tico versus Din\^amico}

% A diferença entre est\ah tico e din\^amico, para \cite{Aho08}, deve-se a pol\'itica de decis\ao que o compilador pode tomar para acusar erros de tipos e opera\co es inv\ah lidas, isto \eh, estaticamente, ou permitir que tais problemas sejam resolvidos em tempo de execu\ca o, de forma din\^amica.

% \cite{Aho08} ainda cita o exemplo de uso da palavra \emph{static} na linguagem Java, que torna um membro de classe pertencente \`a classe, garantindo que n\ao importa quantas c\oh pias da classe existam, haver\ah\ apenas uma c\oh pia do membro.

% \subsection{Escopo, localidade e estado}
% \subsection{Controle de acesso}
% \subsection{Passagem de par\^ametros}
% \subsection{Recursividade e itera\ca o}
