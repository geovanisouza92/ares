
\chapter{Introdu\ca o}
\label{er:introducao}

% OBSERVA\CA O: Ainda ser\ah\ reescrita

Os sistemas de informa\ca o se tornaram parte muito importante nas organiza\co es, passando a ter papel estrat\'egico para o sucesso de seus objetivos.

Com esse papel de destaque, tais sistemas v\^em crescendo em quantidade e complexidade de funcionalidades, desde simples cadastros at\eh\ elaboradas aplica\co es de an\'alise e \footnote{Bussiness Intelligence.}{BI}.

Tal complexidade tem custo na performance e escalabilidade dos sistemas, exigindo que analistas e programadores pensem em estrat\'egias para que seus usu\'arios possam ser produtivos.

Uma das estrat\'egias para melhoria dos sistemas de informa\ca o, \eh\ a otimiza\ca o de c\'odigo, que tem como objetivo reduzir o n\'umero de opera\co es executadas pela m\'aquina sem perda de sentido sem\^antico, ou seja, preservando o objetivo de seus algoritmos.

Dentre as muitas ferramentas de otimiza\ca o existentes, h\'a o \textit{LLVM}, que \eh\ um conjunto de ferramentas para constru\ca o e estudo de compiladores e otimizadores.

Outro fato sobre o LLVM \eh\ sua capacidade de executar otimiza\co es ao longo do tempo de vida do programa, adaptando-se ao seu padr\ao de uso, atrav\'es de duas estrat\'egias:

\begin{enumerate}
\item Um otimizador em tempo de execu\ca o, conhecido como JIT\footnote{\textit{Just-in-Time compilation}. Processo utilizado para compila\ca o de trechos de c\'odigo utilizados com frequ\^encia em c\'odigo nativo de m\'aquina.}\sigla{JIT}{Compilador \emph{Just-In-Time}. T\eh cnica de tradu\ca o de c\oh digo objeto, que transforma \emph{bytecode} em c\oh digo de m\ah quina, por exemplo.};
\item Um otimizador offline, que utiliza informa\co es de profiling da execu\ca o para otimizar o programa bin\'ario.
\end{enumerate}

Tais otimiza\co es tornam-se poss\'iveis pois o LLVM embarca a representa\ca o intermedi\'aria do programa para consulta posterior, podendo gerar programas otimizados em pontos diferentes, adaptando-se melhor aos padr\~oes de uso.

A proposta desse trabalho \eh\ o desenvolvimento de uma linguagem de programa\ca o e uma plataforma virtual baseada na LLVM, que transforme o c\'odigo-fonte na representa\ca o intermedi\ah ria compat\'ivel com a especifica\ca o da \footnote{\textit{Low-Level Virtual Machine Intermediate Representation}. Representa\ca o intermedi\'aria independente de plataforma utilizada pelo LLVM}{LLVM-IR} que ser\'a passada a:

\begin{enumerate}
\item Ambiente de execu\ca o capaz de executar a representa\ca o intermedi\'aria, sem compil\'a-la para um processador espec\'ifico;
\item Backend de gera\ca o de c\'odigo nativo de m\'aquina voltado para uma arquitetura de processador espec\'ifica.
\end{enumerate}
