
% \usepackage{ucs}
\usepackage[utf8]{inputenc} % Dá suporte para caracteres especiais como acentos e cedilha
\usepackage[brazil]{babel} % Gera datas e nomes em português com estilo brasileiro
% \usepackage{supertabular}
\usepackage[hyperindex=false,pdftex]{hyperref} % Permite a criação de hyperlink no documento, como os links usados na referência
\usepackage[alf,abnt-emphasize=bf,abnt-full-initials=yes,abnt-thesis-year=final,abnt-doi=link,abnt-url-package=hyperref]{abntcite} % Define o estilo de referência bibliográfica
\usepackage{tabela-simbolos}
\usepackage[pdftex]{graphicx} % Permite a utiliza\cc ão de imagens no documento
\usepackage[small]{caption} % Define as legendas das figuras com fontes menores do que o texto
\usepackage{pslatex} % Define que o formato da letra será Times New Roman
\usepackage{epigraph} % Permite a cria\cc ão de epígrafes
\usepackage{setspace} % Permite a defini\cc ão de espa\cc amento entre linhas
\usepackage[top=3cm,left=3cm,right=2cm,bottom=2cm]{geometry} % Define as margens da folha

\setcounter{secnumdepth}{3} % Até três subsubsections numeradas
\setcounter{tocdepth}{3} % Até três subsubsections numeradas
\setlength{\parindent}{1.5cm} % Define o recuo da primeira linha dos parágrafos para 1.5 cm
\onehalfspacing % Define o espa\cc amento de 1.5cm entre linhas

% \renewcommand{\rmdefault}{phv} % Fonte Arial
% \renewcommand{\sfdefault}{phv} % Fonte Arial

\renewcommand{\ABNTchapterfont}{\bfseries} % Define a fonte do \chapter
\renewcommand{\ABNTchaptersize}{\large} % Define o tamanho da fonte do \chapter
\renewcommand{\ABNTsectionfontsize}{\large} % Define o tamanho da fonte da \section
\renewcommand{\ABNTsubsectionfontsize}{\large} % Define o tamanho da fonte do \subsection
\renewcommand{\ABNTsubsubsectionfontsize}{\large} % Define o tamanho da fonte do \subsubsection
\renewcommand{\ABNTbibliographyname}{REFER\^ENCIAS BIBLIOGR\'AFICAS} % Modifica o título gerado pelo \bibliographys

\newcommand{\cc}{{\c c}}
\newcommand{\CC}{{\c C}}
\newcommand{\ca}{\cc\~a}
\newcommand{\CA}{\CC\~A}
\newcommand{\co}{\cc\~o}
\newcommand{\CO}{\CC\~O}
