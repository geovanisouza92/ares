% Arquivo de configuração dos arquivos

\usepackage[utf8]{inputenc} % Dá suporte para caracteres especiais como acentos e cedilha
\usepackage[brazil]{babel} % Gera datas e nomes em português com estilo brasileiro
\usepackage[overload]{textcase} % Comandos para alteração da caixa do texto
\usepackage[hyperindex=false,pdftex]{hyperref} % Permite a criação de hyperlink no documento, como os links usados na referência
\usepackage[alf,abnt-emphasize=bf,abnt-full-initials=yes,abnt-thesis-year=final,abnt-doi=link,abnt-url-package=hyperref]{abntcite} % Define o estilo de referência bibliográfica
\usepackage{tabela-simbolos}
\usepackage[usenames,dvipsnames]{color}
\usepackage[pdftex]{graphicx} % Permite a utiliza\cc ão de imagens no documento
\usepackage[small]{caption} % Define as legendas das figuras com fontes menores do que o texto
\usepackage{pslatex} % Define que o formato da letra será Times New Roman
\usepackage{epigraph} % Permite a cria\cc ão de epígrafes
\usepackage{setspace} % Permite a defini\cc ão de espa\cc amento entre linhas
\usepackage[top=3cm,left=3cm,right=2cm,bottom=2cm]{geometry} % Define as margens da folha
\usepackage{listings} % Pacote para exibição de trechos de código
\lstset{
  language=Ruby, % Ruby é a sintaxe mais próxima da linguagem do protótipo
  basicstyle=\footnotesize, % Tamanho da fonte dos trechos
  numbers=left, % Numeração das linhas
  numberstyle=\tiny\color{Gray}, % Estilo de numerações
  stepnumber=5, % Quantidade de linhas para aparecer a numeração; Se 1, cada linha é numerada
%
  numbersep=7pt, % Separação entre as numerações e o código
  backgroundcolor=\color{White},
  showspaces=false,
  showstringspaces=false,
  showtabs=false,
  % frame=single,
  rulecolor=\color{Gray},
  tabsize=2,
  captionpos=b, % Ajusta a legenda para a parte de baixo (bottom)
  breaklines=true, % Word-wrap
  breakatwhitespace=false, % Se as quebras ocorrem somente nos espaçamentos
  title=\lstname,
%
  keywordstyle=\color{OliveGreen}\bfseries,
  commentstyle=\color{Gray},
  stringstyle=\color{Bittersweet},
  morestring=[b]",
  morekeywords={
    abstract,
    after,
    async,
    asc,
    before,
    between,
    break,
    by,
    case,
    const,
    desc,
    elif,
    ensure,
    event,
    false,
    from,
    get,
    group,
    has,
    implies,
    import,
    invariants,
    in,
    join,
    left,
    new,
    null,
    on,
    order,
    retry,
    right,
    sealed,
    select,
    set,
    signal,
    skip,
    step,
    take,
    then,
    true,
    var,
    when,
    where,
    xor
  }
}

\setcounter{secnumdepth}{3} % Até três subsubsections numeradas
\setcounter{tocdepth}{3} % Até três subsubsections numeradas
\setlength{\parindent}{1.5cm} % Define o recuo da primeira linha dos parágrafos para 1.5 cm

% \renewcommand{\rmdefault}{phv} % Fonte Arial
% \renewcommand{\sfdefault}{phv} % Fonte Arial

\renewcommand{\ABNTchapterfont}{\bfseries} % Define a fonte do \chapter
\renewcommand{\ABNTchaptersize}{\large} % Define o tamanho da fonte do \chapter
\renewcommand{\ABNTsectionfontsize}{\large} % Define o tamanho da fonte da \section
\renewcommand{\ABNTsubsectionfontsize}{\large} % Define o tamanho da fonte do \subsection
\renewcommand{\ABNTsubsubsectionfontsize}{\large} % Define o tamanho da fonte do \subsubsection
\renewcommand{\ABNTbibliographyname}{REFER\^ENCIAS BIBLIOGR\'AFICAS} % Modifica o título gerado pelo \bibliographys
\renewcommand{\listofabreviationsname}{LISTA DE SIGLAS} % Modifica o nome da lista de siglas
\renewcommand{\contentsname}{SUM\'ARIO} % Modifica o nome do sumário
\renewcommand{\lstlistingname}{Listagem}
\renewcommand{\lstlistlistingname}{LISTA DE FRAGMENTOS DE C\'ODIGO}

\newcommand{\cc}{{\c c}}
\newcommand{\CC}{{\c C}}
\newcommand{\ca}{\cc\~a}
\newcommand{\CA}{\CC\~A}
\newcommand{\co}{\cc\~o}
\newcommand{\CO}{\CC\~O}
\newcommand{\ao}{\~ao }
\newcommand{\AO}{\~AO }
\newcommand{\ah}{\'a}
\newcommand{\eh}{\'e}
\newcommand{\ih}{\'i}
\newcommand{\oh}{\'o}
\newcommand{\uh}{\'u}
\newcommand{\AH}{\'A}
\newcommand{\EH}{\'E}
\newcommand{\IH}{\'I}
\newcommand{\OH}{\'O}
\newcommand{\UH}{\'U}
