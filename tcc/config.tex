
\usepackage{ucs}
\usepackage[utf8]{inputenc} % Dá suporte para caracteres especiais como acentos e cedilha
\usepackage[brazil]{babel} % Gera datas e nomes em português com estilo brasileiro
\usepackage{hyperref} % Permite a cria{\c c}ão de hyperlink no documento, como os links usados na referência
\usepackage{url}
\usepackage[alf, abnt-emphasize=bf, abnt-full-initials=yes, abnt-thesis-year=final, key="x"]{abntcite} % Define o estilo de referência bibliográfica
\usepackage{graphicx} % Permite a utiliza{\c c}ão de imagens no documento
\usepackage[small]{caption} % Define as legendas das figuras com fontes menores do que o texto
\usepackage{pslatex} % Define que o formato da letra será Times New Roman
\usepackage{epigraph} % Permite a cria{\c c}ão de epígrafes
\usepackage{setspace} % Permite a defini{\c c}ão de espa{\c c}amento entre linhas
\usepackage[top=3cm, left=3cm, right=2cm, bottom=2cm]{geometry} % Define as margens da folha

\setcounter{secnumdepth}{3} % Até três subsubsections numeradas
\setcounter{tocdepth}{3} % Até trẽs subsubsections numeradas
\setlength{\parindent}{1.5cm} % Define o recuo da primeira linha dos parágrafos para 1.5 cm
\onehalfspacing % Define o espa{\c c}amento de 1.5cm entre linhas

% \renewcommand{\rmdefault}{phv} % Fonte Arial
% \renewcommand{\sfdefault}{phv} % Fonte Arial

\renewcommand{\ABNTchapterfont}{\bfseries} % Define a fonte do \chapter
\renewcommand{\ABNTchaptersize}{\large} % Define o tamanho da fonte do \chapter
\renewcommand{\ABNTsectionfontsize}{\large} % Define o tamanho da fonte da \section
\renewcommand{\ABNTsubsectionfontsize}{\large} % Define o tamanho da fonte do \subsection
\renewcommand{\ABNTsubsubsectionfontsize}{\large} % Define o tamanho da fonte do \subsubsection
\renewcommand{\ABNTbibliographyname}{REFER\^ENCIAS BIBLIOGR\'AFICAS} % Modifica o título gerado pelo \bibliographys
