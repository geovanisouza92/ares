% Agradecimentos

\label{tcc:agradecimentos}

\begin{center}
\textbf{AGRADECIMENTOS}
\end{center}

Primeiro, agradecer a Deus, que o saudoso Prof.º Louren\cc o chamava de "O Grande Poeta", pois nos d\ah\ a intelig\^encia necess\ah ria para criarmos todas as coisas com nossas pr\oh prias m\ao s.

Gostaria de dedicar este trabalho \`as pessoas que julgo serem as mais importantes de minha vida, at\'e hoje, pois acreditaram na minha capacidade e nos meus sonhos.

Ao homem que se esfor\cc ou, de sol a sol, para dar-me as oportunidades que ele n\ao\ teve. Ao homem que sempre incentivou meu esfor\cc o e meus estudos, mesmo e principalmente quando n\ao\ entendia muito sobre os assuntos pelos quais eu me interessava, mas acreditava nos meus sonhos. O homem que nunca abandonou o dever de homem, de modelo, de amigo, de colega, de confidente e acima de tudo, de meu pai Jos\eh . Espero que ele entenda isso como um presente de agradecimento por tudo que me deu.

Dedico tamb\eh m ao meu orientador, Prof.º \Orientador, especialmente pois, assim como meu pai, acreditou na minha convic\ca o de tornar essa ideia poss\ih vel, al\eh m \eh\ claro, depois de muita insist\^encia de minha parte.

\`As mulheres, que me ensinaram al\eh m dos conhecimentos que me trouxeram aqui, mas que a
sempre fizeram quest\~ao de me lembrar que eu tinha um grande potencial. Entre elas, Prof.ª Miriam Fajardo, Prof.ª M\ah rcia Fajardo, Prof.ª Shirley Ranieri, entre tantas outras que cometo a indec\^encia de n\ao\ citar.

Devo, como obriga\ca o moral, dedicar este trabalho \`aquela que me trouxe ao mundo: Minha m\ae\ Z\eh lia. Especialmente pois acredito estar t\ao\ orgulhosa de mim, quanto estou hoje em apresentar este trabalho.

Finalmente, dedico este trabalho a todos os que vierem a l\^e-lo, na esperan\c{c}a de que se inspirem, critiquem, concordem, discordem, mas acima de tudo, movam-se da in\eh rcia para produzir mais do que eu.
